% This part is 30 marks and should not exceed two pages including figures.

\subsection{Introduction \& Background}



\subsection{Method}

The gem5 simulation was performed for only the X86 architecture, for each of the 
susan and CRC32 benchmark programs. 
In every simulation, the size of the instruction cache was set to 8kB and the size 
of the data cache was set to 16kB.
For each of these, the l1d\_assoc (data cache association) argument was iterated 
through the set [1 2 4 8 16 32 64] and the cacheline\_size (cache line size) argument 
was iterated through the set [16 32 64 128].

This produced a dataset of 56 individual simulations with varying cache line sizes and 
associativities.
Key values from the results of each of these simulations, such as number of cycles, 
number of instructions, and miss rate were scraped from the output files and assembled 
in table format.

\subsection{Results \& Analysis}

\subsubsection{Cache Associativity}

From the results, it can be seen that increasing the associativity of the cache from 
1 to 2 greatly improves the miss rate of the processor. After that, increasing the 
associativity any further has little effect on the overall performance.

\begin{figure}
    \centering
    \includestandalone[width=.6\textwidth]{graphs/partb-assoc}
    \caption{A comparison of the miss rate for different cache associativities}
    \label{fig:partb-assoc}
\end{figure}

It can be seen that increasing the associativity can nearly completely eliminate 
cache misses when running the CRC32 benchmark, however the susan benchmark sees a near 
constant 0.02% miss rate from an associativity of 2 onwards.

\subsubsection{Cache Line Size}

From the results, it can be seen that increasing the cache line size from 16 to 32 
improves performance for the susan benchmark, but decreases performance for the CRC32 
benchmark. Cache line sizes of 64 and 128 only increase the cache miss rate further for 
both benchmark programs.

\begin{figure}
    \centering
    \includestandalone[width=.6\textwidth]{graphs/partb-cacheln}
    \caption{A comparison of the miss rate for different cache line sizes}
    \label{fig:partb-cacheln}
\end{figure}


\subsection{Conclusion}