% This part is 30 marks and should not exceed two pages including figures.
%
% % REFS
% \cite{Crawford2002} This is a description of the pipeline of an i486 cpu. It's useful
%                     if you want to give a real life example
% \cite{Dwyer1992}    This is a good general citation.
%



\subsection{Introduction}

\begin{itemize}
\item pipeline stalls
\item how O3 reduces stalls
\end{itemize}


\subsection{Method}

The CPU Model from `TimingSimpleCPU' to `Deriv03CPU'.

\begin{lstlisting}[
    language=beef,
    caption={The command used to invoke the \texttt{gem5} simulation for part C.}
]
./gem5/build/X86/gem5.opt -d out_crc_x86 gem5/configs/example/se.py --caches --l1d_size=16kB --l1i_size=8kB --l1d_assoc=64 --cacheline_size=128 --cpu-type=DerivO3CPU -c mibench/telecomm/CRC32/crc_x86 -o mibench/telecomm/adpcm/data/large.pcm
./gem5/build/X86/gem5.opt -d out_susan_x86 gem5/configs/example/se.py --caches --l1d_size=16kB --l1i_size=8kB --l1d_assoc=8 --cacheline_size=128 --cpu-type=DerivO3CPU -c mibench/automotive/susan/susan_x86 -o "mibench/automotive/susan/input_large.pgm output_large.smoothing.pgm -s"
\end{lstlisting}


\subsection{Results}

\begin{itemize}
\item CPI < 1
\end{itemize}


\begin{figure}[H]
    \centering
    \includestandalone[width=.6\textwidth]{graphs/partc-cpi}
    \caption{
        A comparison of two CPUs one with and one without Out of Order
    execution, using CPI as a performance indicator.
    }
    \label{fig:partc-cpi}
\end{figure}
