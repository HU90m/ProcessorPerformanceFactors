% This part is 30 marks and should not exceed two pages including figures.
%
% % REFS
% \cite{Crawford2002} This is a description of the pipeline of an i486 cpu. It's useful
%                     if you want to give a real life example
% \cite{Dwyer1992}    This is a good general citation.
%



\subsection{Introduction}
So far, the `TimingSimpleCPU' has been used for simulations. For each cycle,
this CPU `waits' for the instruction to propagate through all the stages of the
execution path: Instruction Fetch (IF), Instruction Decode (ID), Execute (EX),
Memory Access, (MEM), and Register Write Back (WB).

Pipelining can be was devised to allow higher CPU through put. It does this by
segmenting the execution path into the five stages mentioned above. This five
stage `pipeline' waits for each instruction to complete the stage it is
currently on before moving along to the next stage.  This is far more economical
because the earlier stages aren't idle as the later stages complete the
execution of the instruction. Instead they are filled with the next instruction.
The critical path of the pipeline stages is far shorter than the critical path
of them combined, allowing for a far greater clock speed.  This allows a CPU to
achieve a higher throughput.

However, the higher throughput is hindered by pipeline stalls. This is where an
instruction requires the result of a prior instruction to be executed, and so
has to wait until the prior instruction completes the final WB stage.  The
result of this is the entire pipeline being halted until this instruction can be
executed.

Out of Order execution mitigates this problem by filling these halted and
otherwise unused instruction cycles, which is done by reordering the
instructions.  The next instructions of a program are held in a queue.  Each
instruction continues be held until all its operands are available.  In the
pipeline stall situation previously discussed, a different instruction would be
executed before the instruction that stalled the pipeline. This allows the
instruction it is waiting for to complete the Write Back stage before it reaches
the execution stage. Thus, instead of the pipeline waiting idle for the command
until the instruction can run, a different instruction is run.  The result being
better hardware utilisation, and so faster program run times.  This approach is
effective enough to produce cycles per instruction values of below one.
\cite{Dwyer1992}

Sadly, it is not all fine and dandy. Out of order execution demands a good
deal of chip area to store the next instructions and determine a more optimal
ordering. With this extra area, also comes an increased power requirements.

\subsection{Method}

The x86 architecture with a cache line size of 128 was found to be the best
performer of part B, and so was selected to be the control in the part C test.
Both CRC and SUSAN were rerun changing only the CPU model from `TimingSimpleCPU'
to `DerivO3CPU'.

Unlike `TimingSimpleCPU', the `DerivO3CPU' model employs both piplining and out
of order execution. A comparison of the CPI for each of these CPU models running
with all other factors provides a good demonstration of the improvements
afforded by out of order execution.

\begin{lstlisting}[
    language=beef,
    caption={The two commands used to invoke the \texttt{gem5} simulation for part C.}
]
./gem5/build/X86/gem5.opt -d out_crc_x86 gem5/configs/example/se.py --caches --l1d_size=16kB --l1i_size=8kB --l1d_assoc=64 --cacheline_size=128 --cpu-type=DerivO3CPU -c mibench/telecomm/CRC32/crc_x86 -o mibench/telecomm/adpcm/data/large.pcm
./gem5/build/X86/gem5.opt -d out_susan_x86 gem5/configs/example/se.py --caches --l1d_size=16kB --l1i_size=8kB --l1d_assoc=8 --cacheline_size=128 --cpu-type=DerivO3CPU -c mibench/automotive/susan/susan_x86 -o "mibench/automotive/susan/input_large.pgm output_large.smoothing.pgm -s"
\end{lstlisting}


\subsection{Results}
Figure \label{fig:partc-cpi} shows well the huge improvement in CPI that can be
obtained through the implementation of out of order execution with pipelining.
Of particular note is the reduction of CPI to a value well below one for both
benchmarks. A CPI value below one is impossible without the implementation of
pipelining.
The extent to which the CPI values have decreased verbosely illustrates
uneconomical use of hardware in the simple CPU as stages lie idle between
instructions.

Another interesting point of note is the difference in improvement between the
two benchmarks. Out of order execution likely had a greater effect on CRC than
SUSAN, due to the SUSAN benchmark having a greater program complexity. This
greater program complexity results in an increased complexity in finding a more
optimal instruction ordering, due to more complicated interdependencies between
instructions. Thus CRC was able to be better reordered to maximise hardware
utilisation.

\begin{figure}
    \centering
    \includestandalone[width=.6\textwidth]{graphs/partc-cpi}
    \caption{
        A comparison of two CPUs, one with and one without Out of Order
    execution. CPI is used as a performance indicator.
    }
    \label{fig:partc-cpi}
\end{figure}
