% This part is 40 marks and should not exceed two pages including figures.

\subsection{Introduction \& Background}



\subsection{Method}

The gem5 simulation was performed for both the ARM and X86 architectures, for
each of the susan and CRC32 benchmark programs. For each of these, the l1i\_size 
(instruction cache size) argument was iterated through the set [2kB 4kB 8kB 16kB 32kB]
and the l1d\_size (data cache size) argument was iterated through the set [2kB 4kB 8kB 
16kB 32kB 64kB].

This produced a dataset of 120 individual simulations with varying data and instruction
cache sizes.
Key values from the results of each of these simulations, such as number of cycles, 
number of instructions, and miss rate were scraped from the output files and assembled 
in table format.

\subsection{Results \& Analysis}

\subsubsection{Data Cache Size}

From the results, it can be seen that increasing the size of the data cache from 2kB to 
4kB greatly improves the performance of the processor, however the gains rapidly 
decrease for larger sizes of data cache.

The X86 chip running the CRC32 benchmark program sees the greatest improvements in 
performance.

\begin{figure}[H]
    \centering
    \includestandalone[width=.6\textwidth]{graphs/parta-l1d}
    \caption{A comparison of the CPI results for different sizes of data cache}
    \label{fig:parta-l1d}
\end{figure}

\subsubsection{Instruction Cache Size}

From the results, it can be seen that increasing the size of the instruction cache 
slightly improves the performance of the processor.

% We need to mention some general introductory things here, like our test
% setup[s], and what simulator settings etc. we used :)

\begin{figure}[H]
    \centering
    \includestandalone[width=.6\textwidth]{graphs/parta-l1i}
    \caption{A comparison of the CPI results for different sizes of instruction cache}
    \label{fig:parta-l1i}
\end{figure}


\subsection{Conclusion}
