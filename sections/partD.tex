% This part is 50 marks and should not exceed two pages including figures.

\subsection{Introduction \& Background}
% TODO: REFS

When a pipelined CPU comes across a branch instruction, it must normally
block the pipeline until the outcome of the branch is known, so that the
correct instructions are fetched. This however is slow and inefficient -
the earlier parts of the pipeline must sit around idle, while the later
ones figure out which instructions are next.

A simple solution to this would be to guess the outcome of the branch
instruction randomly, and assume the outcome until the assumption is
proven incorrect.
The pipeline would fill up with instructions that have a \(50\%\) chance
of being the correct ones, and if they are, the contents of the pipeline
can simply be discarded.
This strategy would negate the branch penalty half of the time.

A more advanced approach is branch prediction.
This technique is used by CPUs to improve their guess as to the outcome of
a branch instruction.
Based on information already availiable, the CPU must make a decision about
the most likely outcome of an upcoming branch instruction. The decision can
then be used as an assumption to continue filling the pipe with instructions,
hopefully negating the branch penalty more than half of the time.

\subsection{Techniques}

\subsubsection{Bimodal Predictor}
% TODO: REFS

A bimodal predictor is a very simple type of predictor.
The predictor tags every branch instruction with a two bit counter.
Whenever the particular branch is taken, the counter is incremented,
and whenever it is not taken, the counter is decremented.
The counter does not overflow.
The predictor predicts that a branch will be taken if its counter
is \(10\) or \(11\), and it predicts that a branch will not be taken
if its counter is \(00\) or \(01\).

\subsubsection{TAGE Predictor}
% Refs:
%    \cite{Seznec2006} a-case-for-tage.pdf
%    \cite{Seznec2007} the-l-tage-branch-predictor.pdf
%
% The former was the paper introducing TAGE, and the latter
%  expands on it!

The TAGE predictor, or the \textbf{Ta}gged \textbf{Ge}ometric history
length branch predictor is a more advanced technique for branch
prediction.

This predictor relies on the assumption that if the same sequence of
branches happens repeatedly. then the result of the next branch will often
be the same.
Furthermore, this predictor relies on the assumption that the longer
the length of matching past branch history, the higher
the probability of matching future branches will be.\cite{Seznec2006}

A series of tagged history predictors are used, each using a geometrically
increasing history length - i.e. of the form
\(L_i = \left\lceil \alpha^{i-1} \cdot L_0 \right\rceil\).
If any of the predictors recognize the current branch history,
then the prediction with the longest history is used.
But otherwise, a simpler `base' predictor is used, for example a Bimodal
predictor.\cite{Seznec2007}

\subsubsection{Perceptron Predictor}

\subsection{Simulation}

\subsection{Results}

\begin{figure}
    \centering
    \includestandalone[width=.6\textwidth]{graphs/partd-cpi}
    \caption{A comparison of our CPI results for different branch predictors}
    \label{fig:partd-cpi}
\end{figure}


